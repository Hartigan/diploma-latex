\newcommand{\byr}{Br}

\section{ТЕХНИКО-ЭКОНОМИЧЕСКОЕ ОБОСНОВАНИЕ ЭФФЕКТИВНОСТИ РАЗРАБОТКИ И ИСПОЛЬЗОВАНИЯ МНОГОСЛОЙНЫХ ПЕРЦЕПТРОНОВ ДЛЯ СЖАТИЯ ИЗОБРАЖЕНИЙ}

\subsection{Характеристика программного продукта}

В рамках дипломной работы поставлена цель найти способ сжимать данные с использованием многослойных перцептронов. В данной работе проведено исследование в области сжатия графической информации. Произведены оценки скорости, степени сжатия и качества полученных изображений. Программный продукт позволит продемонстрировать результаты использования нейронных сетей для сжатия аналоговой информации и оценить качество сжатия данных методом «сжатия с потерями».

В данном разделе будут рассмотрены затраты на разработку и использование приложения основанное на использовании нейронных сетей. Данное прилолжение позволит:
\begin{enumerate}

  \item эффективно сжимать изображения

  \item изучить возможности использования нейронных сетей для сжатия графической информации

\end{enumerate}

Экономическая целесообразность инвестиций в разработку и использование программного продукта осуществляется на основе расчета и оценки следующих показателей:

\begin{itemize}

  \item чистая  дисконтированная стоимость ($ \text{Ч}_\text{дд} $);
  
  \item срок окупаемости инвестиций ($ \text{Т}_\text{ок} $);
  
  \item рентабельность инвестиций ($ \text{Р}_\text{и} $).
  
\end{itemize}

Полученные результаты позволят строить программные комплексы, использование которых, улучшит эффективность сжатия данных методом «сжатия с потерями».

Для оценки экономической эффективности инвестиционного проекта по разработке и внедрению программного продукта необходимо рассчитать:

\begin{enumerate}

  \item Результат ($ \text{Р} $), получаемый от использования программного продукта;
  
  \item Затраты (инвестиции), необходимые для разработки программного продукта;
  
  \item Показатели эффективности инвестиционного проекта по производству программного продукта.
  
\end{enumerate}

\subsection{Расчет стоимостной оценки затрат}

Общие капитальные вложения ($ \text{K}_\text{о} $) заказчика (потребителя), связанные с приобретением, внедрением и использованием ПС, рассчитываются по формуле:
\begin{equation}
  \label{eq:econ:total_program_size_corrected}
  \text{К}_{\text{о}} = \text{К}_{\text{пр}} + \text{К}_{\text{ос}}\text{.}
\end{equation}
\begin{explanation}
где & $ \text{К}_{\text{пр}} $ & затраты пользователя на приобретение ПС по отпускной цене разработчика с учетом стоимости услуг по эксплуатации и сопровождению (тыс.руб.); \\
    & $ \text{К}_{\text{ос}} $ & затраты пользователя на освоение ПС (тыс. руб.).
\end{explanation}

\subsubsection{Расчет стоимостной оценки затрат}
Основная заработная плата исполнителей на наш программный продукт рассчитывается по формуле:
\begin{equation}
  \label{eq:econ:spendings}
  \text{З}_{\text{о}} = \sum_{i=1}^n \text{Т}_{\text{чi}}\text{Т}_{\text{ч}}\text{Ф}_{\text{п}}\text{К.}
\end{equation}
\begin{explanation}
где & $ \text{n} $ & количество исполнителей, занятых разработкой программного продукта; \\
    & $ \text{Т}_{\text{чi}} $ & часовая тарифная ставка i-го исполнителя (тыс. руб.); \\
    & $ \text{Ф}_{\text{п}} $ & плановый фонд рабочего времени i-го исполнителя (дн.); \\
    & $ \text{Т}_{\text{ч}} $ & количество часов работы в день (ч). \\
    & $ \text{К} $ & коэффициэнт премирования.
\end{explanation}

Коэффициент премирования 1,6. Для расчета заработной платы месячная тарифная ставка 1-го разряда на предприятии установлено на уровне одного миллиона восьмиста шестидесяти тысяч белорусских рублей.

\begin{table}[h]
\caption{Расчёт заработной платы}
\begin{tabular}{|l|l|l|l|l|l|}
\hline
\begin{tabular}[c]{@{}l@{}}Категория \\ исполнителя\end{tabular} & Разряд & \begin{tabular}[c]{@{}l@{}}Тарифный\\ коэффициент\end{tabular} & \begin{tabular}[c]{@{}l@{}}Часовая \\ тарифная \\ ставка, \\ тыс. руб.\end{tabular} & \begin{tabular}[c]{@{}l@{}}Трудоё- \\мкость, \\ дн.\end{tabular} & \begin{tabular}[c]{@{}l@{}}Основная\\ заработная \\ плата, \\ тыс. руб.\end{tabular} \\ \hline
\begin{tabular}[c]{@{}l@{}}программист 1-ой \\ категории\end{tabular} & 14 & 3.66 & 28.5 & 30 & 6760 \\ \hline
\begin{tabular}[c]{@{}l@{}}руководитель \\ проекта\end{tabular} & 16 & 3 & 32.1 & 30 & 7704 \\ \hline
\begin{tabular}[c]{@{}l@{}}Итого с \\ премией \\ (60\%)\end{tabular}   & - & - & - & - & 14544 \\ \hline
\end{tabular}
\end{table}


Дополнительная заработная плата на наш программный продукт ($\text{З}_\text{д}$) включает выплаты, предусмотренные законодательством о труде (оплата отпусков, льготных часов, времени выполнения государственных обязанностей и других выплат, не связанных с основной деятельностью исполнителей), и определяется по нормативу в процентах к основной заработной плате:
\begin{equation}
  \text{З}_{\text{д}} = \frac{\text{З}_{\text{o}}\text{Н}_{\text{д}}}{\num{100}\%}{\,,}
\end{equation}
\begin{explanation}
где & $ \text{З}_{\text{д}} $ & дополнительная заработная плата исполнителей (тыс. руб.) \\
    & $ \text{Н}_{\text{д}} $ & норматив дополнительной заработной платы равный 20\%.
\end{explanation}

\begin{equation}
  \text{З}_{\text{д}} = \frac{\num{14544}\cdot \num{20}\%}{\num{100}\%} = \num{2908}\text{ тыс.руб.} \text{\,,}
\end{equation}

Отчисления в фонд социальной защиты населения и обязтельное страхование ($ \text{З}_{\text{сз}} $) определяются в соответствии с действующими законодательными актами по нормативу в процентном отношении к фонду основной и дополнительной зарплаты исполнителей, определенной по нормативу, установленному в целом по организации:
\begin{equation}
\text{З}_{\text{сз}} = \frac{(\text{З}_{\text{о}} + \text{З}_{\text{д}})\text{Н}_{\text{сз}}}{\num{100}\%} \text{\,,}
\end{equation}

\begin{explanation}
где & $ \text{Н}_{\text{сз}} $ & норматив отчислений в фонд социальной защиты населения  и на обязательное страхование (34 + 0,6\%)
\end{explanation}

\begin{equation}
  \text{З}_{\text{сз}} = \frac{(\num{14544} + \num{2908})\cdot \num{34.6}\%}{\num{100}\%} = \num{6038.7}\text{ тыс.руб.} \text{\,,}
\end{equation}

Расходы по статье «Машинное время» ($ \text{Р}_\text{м} $) включают оплату машинного времени, необходимого для разработки и отладки программного продукта, которое определяется по нормативам (в машино-часах) на 100 строк исходного кода ($ \text{H}_\text{мв}$) машинного времени, и определяются по формуле:

\begin{equation}
\text{Р}_{\text{м}} = \text{Ц}_{\text{м}} + \text{Т}_{\text{пр}} \text{\,,}
\end{equation}

\begin{explanation}
где & $ \text{Ц}_{\text{м}}$ & цена одного машино-часа. Рыночная стоимость машино-часа компьютера со всеми необходимым оборудованием (10 тыс. руб. / ч); \\
    & $ \text{Т}_{\text{пр}} $ & время работы над программным продуктом (25дн*8ч = 200ч).
\end{explanation}

\begin{equation}
\text{Р}_{\text{м}} = \num{10}\cdot \num{200} = \num{2000}\text{ тыс.руб.} \text{\,,}
\end{equation}

Расходы по статье «Научные командировки» ($ \text{Р}_\text{нк} $) на програмнное средство определяются по формуле:
\begin{equation}
\text{Р}_{\text{нк}} = \frac{\text{З}_\text{о}\text{Н}_\text{рнк}}{100\%} \text{\,,}
\end{equation}

\begin{explanation}
где & $ \text{Н}_\text{рнк} $ & норматив расходов на командировки в целом по организации (\%). Норматив на командировки - 12 \% от основной заработной платы.
\end{explanation}

\begin{equation}
\text{Р}_{\text{нк}} = \frac{\num{14544}\cdot\num{12}}{\num{100}\%}=\num{1745,3}\text{ тыс.руб.} \text{\,,}
\end{equation}

Расходы по статье «Прочие затраты» ($ \text{П}_\text{з} $) на программное средство включают затраты на приобретение и подготовку специальной научной и технической информации и специальной литературы. И определяются по формуле:

\begin{equation}
\text{П}_{\text{з}} = \frac{\text{З}_\text{о}\text{Н}_\text{пз}}{100\%} \text{\,,}
\end{equation}

\begin{explanation}
где & $ \text{Н}_\text{пз} $ & норматив прочих затрат в целом по организации равен 50\%.
\end{explanation}

\begin{equation}
\text{П}_{\text{з}} = \frac{\num{14544}\cdot\num{22}\%}{\num{100}\%}=\num{3199,7}\text{ тыс.руб.} \text{\,,}
\end{equation}

Затраты по статье «Накладные расходы» ($ \text{Р}_\text{н} $), связанные с необходимостью содержания аппарата управления, вспомогательных хозяйств и опытных (экспериментальных) производств, а также с расходами на общехозяйственные нужды ($ \text{Р}_\text{н} $), и определяют по формуле:

\begin{equation}
\text{Р}_\text{н} = \frac{\text{З}_\text{о}\text{Н}_\text{рн}}{100\%} \text{\,,}
\end{equation}

\begin{explanation}
где & $ \text{Н}_\text{рн} $ & накладные расходы на программный продукт (тыс. руб.); \\
    & $ \text{Н}_\text{рн} $ & норматив накладных расходов в целом по организации, 50\%.
\end{explanation}

\begin{equation}
\text{P}_{\text{н}} = \frac{\num{14544}\cdot\num{50}\%}{\num{100}\%}=\num{7272}\text{ тыс.руб.} \text{\,,}
\end{equation}


Общая сумма расходов по смете ($ \text{С}_\text{р} $) на программный продукт рассчитывается по формуле:

\begin{equation}
\text{С}_\text{р} = \text{З}_\text{о} + \text{З}_\text{д} + \text{З}_\text{сз} + \text{Р}_\text{м} + \text{Р}_\text{нк} + \text{П}_\text{з} + \text{Р}_\text{н} \text{\,,}
\end{equation}

\begin{equation}
\begin{gathered}
\text{С}_\text{р} = \num{14544} + \num{2908} + \num{6038.7} + \num{2000} + \num{1745.3} + \\
+ \num{3199.7} + \num{7272} = \num{23163.7} \text{ тыс.руб.} \text{\,,}
\end{gathered}
\end{equation}

Кроме того, организация-разработчик осуществляет затраты на сопровождение и адаптацию программного продукта ($ \text{Р}_\text{са} $), которые определяются по формуле:

\begin{equation}
\text{Р}_\text{са} = \frac{\text{С}_\text{р}\text{Н}_\text{рса}}{\num{100}\%} \text{\,,}
\end{equation}

\begin{explanation}
где & $ \text{Н}_\text{рса} $ & норматив расходов на сопровождение и адаптацию 20\%.
\end{explanation}

\begin{equation}
\text{Р}_\text{са} = \frac{\num{23163.7}\cdot\num{20}\%}{\num{100}\%} = \num{4632.8} \text{\,,}
\end{equation}

Общая сумма расходов на разработку (с затратами на сопровождение и адаптацию) как полная себестоимость программно продукта ($ \text{С}_\text{п}$) определяется по формуле:

\begin{equation}
\text{С}_\text{п} = \text{С}_\text{р} + \text{Р}_\text{са} \text{\,,}
\end{equation}

\begin{equation}
\text{С}_\text{п} = \num{23163.7} + \num{4632.8} = \num{27796.5} \text{\,,}
\end{equation}

Прибыль рассчитывается по формуле:

\begin{equation}
\text{П}_\text{о} = \frac{\text{С}_\text{п}\text{У}_\text{рп}}{\num{100}\%} \text{\,,}
\end{equation}

\begin{explanation}
где & $ \text{П}_\text{о} $ & прибыль от реализации программного продукта заказчику (тыс. руб.);\\
& $ \text{У}_\text{рп} $ & уровень рентабельности программного продукта 20\%;\\
& $ \text{С}_\text{п} $ & себестоимость програмнного продукта (тыс. руб.).
\end{explanation}

\begin{equation}
\text{П}_\text{о} = \frac{\num{27796.5}\cdot\num{20}\%}{\num{100}\%} = \num{5559.3}\text{ тыс.руб.} \text{\,,}
\end{equation}

Прогнозируемая цена нашего программного продукта без налогов ($ \text{Ц}_\text{п}$):

\begin{equation}
\text{Ц}_\text{п} = \text{С}_\text{р} + \text{П}_\text{о} \text{\,,}
\end{equation}

\begin{equation}
\text{Ц}_\text{п} = \num{23163.7} + \num{5559.3} = \num{28723} \text{ тыс.руб.} \text{\,,}
\end{equation}


\subsection{Расчет стоимостной оценки результата}
Результатом ($ \text{Р} $) в сфере использования нашего программного продукта является прирост чистой прибыли и амортизационных отчислений.
\subsubsection{Расчет прироста чистой прибыли}
Прирост чистой прибыли представляет собой экономию затрат на заработную плату и начислений на заработную плату, полученную в результате внедрения программного продукта, составит:

\begin{equation}
\text{Э}_\text{з} = \text{К}_\text{пр}(\text{t}_\text{c}\text{T}_\text{c} - \text{t}_\text{н}\text{T}_\text{н})\text{N}_\text{п}(1+\frac{\text{Н}_\text{др}}{\num{100}\%})(1+\frac{\text{Н}_\text{нпо}}{\num{100}\%}) \text{\,,}
\end{equation}

\begin{explanation}
где & $ \text{N}_\text{п} $ & плановый объем работ по анализу и обработки результатов, сколько раз выполнялись в году (\num(24) раз);\\
    & $ \text{t}_\text{c} $ & трудоемкость выполнения работы до внедрения программного продукта; (24 нормочасов);\\
    & $ \text{t}_\text{п} $ & трудоемкость выполнения работы после внедрения программного продукта; (4 нормочаса);\\
    & $ \text{T}_\text{c} $ & часовая тарифная ставка, соответствующая разряду выполняемых работ до внедрения программного продукта; (35 тыс. руб. /ч)\\
    & $ \text{Т}_\text{п} $ & часовая тарифная ставка, соответствующая разряду выполняемых работ после внедрения программного продукта; (50 тыс. руб. /ч)\\
    & $ \text{К}_\text{пр} $ & коэффициент премий (1.6);\\
    & $ \text{Н}_\text{д} $ & норматив дополнительной заработной платы (10\%);\\
    & $ \text{Н}_\text{по} $ & ставка отчислений в ФСЗН и обязательное страхование (34+0,6\%).
\end{explanation}

\begin{equation}
\text{Э}_\text{з} = \num{1.6}\cdot(\num{24}\cdot\num{35} - \num{4}\cdot\num{50})\cdot\num{24}\cdot{}(1+\frac{\num{10}\%}{\num{100}\%})(1+\frac{\num{34.6}}{\num{100}\%}) = \num{36387.3} \text{ тыс.руб.}\text{\,,}
\end{equation}

Прирост чистой прибыли рассчитывается по формуле:

\begin{equation}
\text{П}_\text{ч} = \sum_{\text{i=1}}^{n}{\text{Э}_\text{i}(1-\frac{\text{Н}_\text{п}}{\num{100}\%})} \text{\,,}
\end{equation}
\begin{explanation}
где & $ \text{n} $ & виды затрат, по которым получена экономия;\\
    & $ \text{Э} $ & сумма экономии, полученная за счет снижения i-ых затрат, тыс. руб.\\
    & $ \text{Н} $ & ставка налога на прибыль,  18\%.
\end{explanation}
\begin{equation}
\text{П}_\text{ч} = \num{36387.3}\cdot(1-\frac{\num{18}\%}{\num{100}\%}) = \num(6549.7) \text{ тыс.руб.}\text{\,,}
\end{equation}



\subsubsection{Расчет прироста амортизационных отчислений}
Амортизационные отчисления являются источником погашения инвестиций в приобретение программного продукта.      Расчет амортизационных отчислений осуществляется по формуле:
\begin{equation}
\text{А} =\frac{\text{Н}_\text{а}\text{И}_\text{об}}{\num{100}\%} \text{\,,}
\end{equation}
\begin{explanation}
где & $ \text{Н}_\text{а} $ & норма амортизации программного продукта 20\%;\\
    & $ \text{И}_\text{об} $ & стоимость программного продукта, тыс. руб.
\end{explanation}
\begin{equation}
\text{А} =\frac{\num{20}\%\cdot\num{28723}}{\num{100}\%}=\num{5744,6} \text{\,,}
\end{equation}


\subsection{Расчет показателей экономической эффективности проекта}
При оценке эффективности инвестиционных проектов необходимо осуществить приведение затрат и результатов, полученных в разные периоды времени, к  расчетному году,  путем умножения затрат и результатов на коэффициент дисконтирования $ \text{a}_\text{t} $, который определяется следующим образом:

\begin{equation}
\text{a}_\text{t} =\frac{\num{1}}{(1+\text{E}_\text{н})^(\text{t}-\text{t}_\text{p})} \text{\,,}
\end{equation}
\begin{explanation}
где & $ \text{E}_\text{н} $ & требуемая норма дисконта, 30\%;\\
    & $ \text{t} $ & порядковый номер года, затраты и результаты которого приводятся к расчетному году;\\
    & $ \text{t}_\text{p} $ & расчетный год, в качестве расчетного года принимается год вложения инвестиций, равный 1.
\end{explanation}

\begin{equation}
    \text{a}_\text{1}=\frac{\num{1}}{(1+\num{0.3})^\text{1-1}} = \num{1} \text{\,;}
\end{equation}
    
\begin{equation}    
    \text{a}_\text{2} =\frac{\num{1}}{(1+\num{0.30})^\text{2-1}} = \num{0.769} \text{\,;}
\end{equation}
    
\begin{equation}     
    \text{a}_\text{3} =\frac{\num{1}}{(1+\num{0.30})^\text{3-1}} = \num{0.591} \text{\,;}
\end{equation}
    
\begin{equation}      
    \text{a}_\text{4} =\frac{\num{1}}{(1+\num{0.30})^\text{4-1}} = \num{0.455} \text{\,;}
\end{equation}
    
\begin{table}[h]
\caption{Экономические результаты работы предприятия}
\begin{tabular}{|p{35mm}|p{10mm}|p{10mm}|p{20mm}|p{20mm}|p{20mm}|p{20mm}|}
\hline
Наименование показателей & Един. измер. & Усл. обоз. & \multicolumn{4}{c|}{По годам использования ПП}     \\ \hline
\multicolumn{1}{|c|}{} & & & 1-ый & 2-ой & 3-ий & 4-ый \\ \hline
1. Прирост чистой прибыли & Тыс. руб. & $\Delta \text{П}_\text{ч} $ & 2729 & 6549.7 & 6549.7 & 6549.7 \\ \hline
2. Прирост автоматизированных отчислений & Тыс. руб. &  $\Delta \text{А} $ & 5744.6 & 5744.6 & 5744.6 & 5744.6   \\ \hline
3. Прирост результата & Тыс. руб. & $ \Delta \text{P}_\text{t} $ & 8473.6 & 12294.3 & 12294.3 & 12294.3 \\ \hline
4. Коэфициент дисконтирования & & $\text{a}_\text{t} $ & 1 & 0.769 & 0.591 & 0.455 \\ \hline
5. Результат с учётом фактора времени & Тыс. руб. & $\text{P}_\text{t}\text{a}_\text{t} $ & 8473.6 & 9454.3 & 7265.9 & 5593.9 \\ \hline
6. Инвестиции & Тыс. руб. &  $ \text{И}_\text{об} $ & 28723 & & & \\ \hline
7. Инвестиции с  учётом фактора времени & Тыс. руб. & $ \text{И}_\text{т}\text{a}_\text{t} $ & 28723 & & & \\ \hline
8. Чистый дисконтированный доход по годам & Тыс. руб. & $\text{ЧДД}_\text{t}$ & -20294.4 & 9454.3 & 7265.9 & 5593.9 \\ \hline
9. ЧДД нарастающим итогом & Тыс. руб. & $\text{ЧДД} $ & -20249.4 & -10795.1 & -3529.2
 & 2064.8 \\ \hline
\end{tabular}
\end{table}

Рассчитаем рентабельность инвестиций ($ \text{Р}_\text{и} $) по формуле:

\begin{equation}
\text{Р}_\text{и} =\frac{\text{П}_\text{чср}}{\text{З}}\cdot\num{100}\% \text{\,,}
\end{equation}
\begin{explanation}
где & $ \text{З} $ & затраты на приобретения нашего программного продукта;\\
    & $ \text{П}_\text{чсp} $ & среднегодовая величина чистой прибыли за расчетный период, тыс. руб., которая определяется по формуле:
\end{explanation}
\begin{equation}
\text{П}_\text{чср} =\frac{\sum_{\text{i=1}}^\text{n}{\text{П}_\text{чi}}}{\text{n}} \text{\,,}
\end{equation}
\begin{explanation}
где & $ \text{П}_\text{чi} $ & чистая прибыль, полученная в году $\text{i}$, тыс. руб.
\end{explanation}

\begin{equation}
\text{П}_\text{чср} =\frac{\num{2729}+\num{6549.7}+\num{6549.7}+\num{6549.7}}{\num{4}}=\num{5594.5}\text{ тыс.руб.} \text{\,,}
\end{equation}

\begin{equation}
\text{Р}_\text{и} =\frac{\num{5594.5}}{\num{28723}}\cdot\num{100}=\num{19.5}\% \text{\,,}
\end{equation}
В результате технико-экономического обоснования инвестиций по производству нового изделия были получены следующие значения показателей их эффективности:

\begin{enumerate}
\item Чистый дисконтированный доход за четыре года производства продукции составит 6549.7 тыс. руб.
\item Все инвестиции окупаются на 4 год
\item Рентабельность инвестиций составляет 19.5\%
\end{enumerate}

Таким образом проведение исследования и разработка программного продукта являются эффективными. Следовательно, инвестирование в разработку и внедрение программы сжатия данных основанной на многослойных перцептронах являетcя целесообразным.