{
  \newgeometry{top=1.25cm,bottom=1.25cm,right=1cm,left=2cm,twoside}
  \thispagestyle{empty}
  \setlength{\parindent}{0em}

  \newcommand{\lineunderscore}{\uline{\hspace*{\fill}}}

  \begin{center}
    Министерство образования Республики Беларусь\\
    Учреждение образования\\
    БЕЛОРУССКИЙ ГОСУДАРСТВЕННЫЙ УНИВЕРСИТЕТ \\
    ИНФОРМАТИКИ И РАДИОЭЛЕКТРОНИКИ\\[1em]


  \begin{minipage}{\textwidth}
    \begin{flushleft}
      \begin{tabular}{ p{0.20\textwidth}p{0.31\textwidth}p{0.20\textwidth}p{0.20\textwidth} @{} }
        Факультет & КСиС & Кафедра & Информатики \\
        Специальность   & 1-31 03 04 & Специализация & 07
      \end{tabular}
    \end{flushleft}
  \end{minipage}\\[1em]

  \begin{minipage}{\textwidth}
    \begin{flushright}
      \begin{tabular}{p{0.40\textwidth}}
        УТВЕРЖДАЮ \\[0.5em]
        \underline{\hspace*{7em}} Н. А. Волорова \\
        <<\underline{\hspace*{4ex}}>> \underline{\hspace*{7em}} 2015 г.
      \end{tabular}
    \end{flushright}
  \end{minipage}\\[1em]

  \textbf{ЗАДАНИЕ} \\
  \textbf{по дипломной работе студента}

  \lineunderscore\uline{Сафонова Анатолия Анатольевича}\lineunderscore \\
  {\small (фамилия, имя, отчество) }

  \end{center}

  1. Тема проекта (работы):
  \uline{Многослойные перцептроны для сжатия изображений}\lineunderscore\\
  \lineunderscore\\
  \lineunderscore\\
  утверждена приказом по университету от \uline{25.03}.2015 г.  \textnumero \uline{502}-с

  \vspace{1em}

  2. Срок сдачи студентом законченной работы: \uline{1 июня 2015 года}\lineunderscore

  \vspace{1em}

  3. Исходные данные к проекту (работе):
  \uline{Платформа .NET Framework 4.5; план проведения исследования; язык программирования С\#; литература по нейронным сетям; литература по методам сжатия данных с потерями.}\lineunderscore\\
  \lineunderscore

  \vspace{1em}

  4. Содержание пояснительной записки (перечень подлежащих разработке вопросов):
  \uline{1 Обзор предметной области}\lineunderscore\\
  \uline{2 Используемые технологии}\lineunderscore\\
  \uline{3 Постановка задачи}\lineunderscore\\
  \uline{4 Разработка алгоритма}\lineunderscore\\
  \uline{5 Архитектура прототипа}\lineunderscore\\
  \uline{6 Анализ полученных данных}\lineunderscore\\
  \uline{7 Обеспечение безопасных условий труда инженера-программиста при проведении исследования многослойных перцептронов для сжатия изображений}\lineunderscore\\
  \uline{8 Технико-экономическое обоснование эффективности разработки и использования многослойных перцептронов для сжатия изображений}\lineunderscore\\
  \uline{Заключение}\lineunderscore\\
  \uline{Список используемых источников}\lineunderscore\\
  \uline{Приложение А Листинг программного средства}\lineunderscore

  \clearpage
  \thispagestyle{empty}

  5. Перечень графического материала (с точным указанием обязательных чертежей):
  \lineunderscore\\
  \lineunderscore\\
  \lineunderscore\\
  \lineunderscore\\
  \lineunderscore\\
  \lineunderscore\\
  \lineunderscore\\
  \lineunderscore

  \vspace{1em}

  6. Содержание задания по технико-экономическому обоснованию:
  \lineunderscore \\
  \uline{Технико-экономическое обоснование эффективности разработки и использования многослойных перцептронов для сжатия изображений}\lineunderscore

  Задание выдал: \hfill{} \uline{\hspace*{6em}} / Т.\,Л.~Слюсарь /

  \vspace{1em}

  7. Содержание задания по охране труда и экологической безопасности, ресурсо- и энергосбережению:
  \lineunderscore \\
  \uline{Обеспечение безопасных условий труда инженера-программиста при проведении исследования многослойных перцептронов для сжатия изображений}\lineunderscore

  Задание выдал:  \hfill{} \uline{\hspace*{6em}} / Т.\,В.~Гордейчук /

  \vfill

  \begin{center}
    КАЛЕНДАРНЫЙ ПЛАН
  \end{center}

  \begin{tabular}{| >{\centering}m{0.04\textwidth}
                  | >{\raggedright}m{0.40\textwidth}
                  | >{\centering}m{0.08\textwidth}
                  | >{\centering}m{0.19\textwidth}
                  | >{\centering\arraybackslash}m{0.16\textwidth}|}
    \hline \textnumero \textnumero \\ п/п & Наименование этапов дипломного проекта (работы) & Объем этапа, \% & Срок выполнения этапов & Примечание \\
    \hline 1 & Анализ предметной области,    &    &                & \\
    \hline   & разработка плана исследования & 5  & 03.02 -- 11.02 & \\
    \hline 2 & Разработка алгоритма          & 10 & 12.02 -- 23.02 & \\
    \hline 3 & Проектирование структуры      &    &                & \\
    \hline   & прототипа                     & 15 & 24.02 -- 10.03 & \\
    \hline 4 & Разработка прототипа          & 30 & 11.03 -- 13.04 & \\
    \hline 5 & Сбор и анализ полученных      &    &                & \\
    \hline   & материалов                    & 30 & 14.04 -- 27.04 & \\
    \hline 6 & Оформление пояснительной      &    &                & \\
    \hline   & записки и графического        &    &                & \\
    \hline   & материала                     & 10 & 28.04 -- 30.05 & \\
    \hline
  \end{tabular}

  \vspace{2em}

  Дата выдачи задания: \uline{\hspace*{6em}} \hspace{2ex} Руководитель \hfill{} \uline{\hspace*{4em}} / Е.\,В.~Кукар /

  \vspace{1em}

  Задание принял к исполнению \hfill{} \uline{\hspace*{4em}} / А.\,А.~Сафонов /

  \restoregeometry
}
