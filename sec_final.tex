\sectioncentered*{Заключение}
\addcontentsline{toc}{chapter}{Заключение}

В данной работе были рассмотрены вопросы использования функционального и реактивного программирования для обработки данных.
Также было разработано подмножество операций реляционной алгебры для динамических данных.
Была разработана библиотека типов и функций в качестве реализации данной алгебры на платформе \dotnet{}.
Разработанная библиотека функций может быть использована при разработке коммерческих продуктов на платформе Microsoft
.NET для реализации взаимодействий между моделями, сервисами, для построенения отзывчивых пользовательских интерфейсов, созданных для получения актуальной информации.
Единственными наиболее близким по функциональности продуктом со схожей областью
применения для платформы Microsoft .NET является библиотека Dynamic Data,
разработанная Roland Pheasant. Отличие данной работы, что операции и функции могут использовать не только события изменения коллекций, но и изменения самих элементов.

В целом разработанная библиотека включает требуемую функциональность, необходимую для практического использования. В библиотеке присутствуют функции для решения ряда задач связанных с применением
реляционной алгебры: объединение, пересечение, вычитание, декартово произведение, выборка, проекция, соединение.
Помимо этого реализованы функции агрегаторы и операции сортировки.

В результате цель работы была достигнута. Было создано программное обеспечение решающие задачи, связанных с применением релационной алгебры и событийной модели на практике.
Было разработан алгоритм и проверена его работоспособность на реальных данных. За рамками проделанной работы остались некоторые специфические вопросы,
например, оптимизация выполнение данных операций, реализация более специфичных функций для работы с данными.
Эти вопросы возникают не во всех практических задачах, но при необходимости разработанная библиотека может быть доработана.
Эти задачи также являются нетривиальными и требуют детального изучения и проработки,
они не рассматривались в данной работе из-за временных ограничений на их исследвание.

В дальнейшем планируется развивать и довести существующее ПО до полноценной библиотеки, способной решать более широкий класс задач,
возникающих в области построения отзывчивых пользовательских интерфейсов.
