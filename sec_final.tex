\sectioncentered*{Заключение}
\addcontentsline{toc}{section}{Заключение}

В данном дипломной работе был рассмотрен вопрос использования многослойных перцептронов для сжатия изображений методом сжатия с потерями.
В рамках дипломной работы был разработан поэтапный алгоритм преобразования графических данных с разными степенями сжатия.
Был разработан прототип, который наглядно демонстрировал работоспособность данного метода.
Были собраны многочисленные статистические данные, произведена качественная оценка получившихся результатов.

В целом были получены удовлетворительные результаты проверенные с помощью методов сравнения изображений с оригиналом,
такими как соотношения сигнал-шум и среднеквадратического отклонения.
Результаты работы реализованных в прототипе функций помогли выделить основные факторы, влияющие на степень сжатия изобращения,
скорость обработки и степень отличия от исходного изображения.
Данный способ удовлетворительно зарекомендовал себя в проведённых тестах.

В результате цель дипломной работы была достигнута.
Было разработан алгоритм и проверена его работоспособность на реальных данных.
Но за рамками рассматриваемой темы осталось еще много других алгоритмов, например использование других типов сетей и методов их обучения.
Алгоритм имеет большой потенциал для его оптимизации путем распараллеливания процессов,
переноса вычислений на графические процессора или возможность задействования многоядерной архитектуры центрального процессора.
Данным алгоритм можно применить для сжатия другой мультимедийной информации, для полного восприятия которой,
человек не имеет достаточного количества биологических ресурсов.
Эти задачи также являются нетривиальными и требуют детального изучения и проработки,
они не рассматривались в данной работе из-за временных ограничений на их исследвание.

В дальнейшем планируется развивать и довести существующее ПО до полноценной библиотеки,
способной решать более широкий класс задач, возникающих в области применения нейронных
сетей и сжатия мультимедийной информации.
