\sectioncentered*{Аннотация}
\thispagestyle{empty}

\begin{center}
  \begin{minipage}{0.82\textwidth}
    на дипломную работу <<Многослойные перцептроны для сжатия изображений>> студента УО <<Белорусский государственный университет информатики и радиоэлектроники>> Сафонова~А.\,А.
  \end{minipage}
\end{center}

\emph{Ключевые слова}: нейронные сети; сжатие с потерями; частота дискретизации; критерии сравнения изображений.

\vspace{4\parsep}

Дипломная работа выполнена на 6 листах формата А1 с пояснительной запиской на~\pageref*{LastPage} страницах, без приложений справочного или информационного характера.
Пояснительная записка включает \total{section}~глав, \totfig{}~рисунков, \tottab{}~таблиц, \toteq{}~формулы, \totref{}~литературный источник.

Целью дипломной работы является разработка алгоритма, пригодного для решения практических задач, возникающих в реальных проектах.%, связанных с обработкой и передачи изображений и другой мультимедийной информации.

Для достижения цели дипломной работы был разработан алгоритм, предназначенный для сжатия графической или любой другой мультимедийной информации.
%Созданные с использованием данного алгоритма библиотеки могут быть использованы в реальных проектах, использующих методы сжатия с потерями для мультемидийной информации.
%В работе приведена библиотека реализующая алгоритм на платформе \dotnet{}.

Во введении производится ознакомление с проблемой, решаемой в дипломной работе.

В первой главе производится обзор предметной области проблемы решаемой в данной дипломной работе.
%Приводятся необходимые теоретические сведения, существующие алгоритмы и предпологаеме способы решения.

Во второй главе производится краткий обзор технологий, использованных для реализации прототипа ПО в рамках дипломной работы.
%Аргументируется выбор языка и платформы для разработки.

В третьей главе производится постановка задачи для данного исследования.
%Описываются рамки исследуемой области и проблемы, с которыми можно столкнуться.

В четвертой главе производится поэтапное описание создания алгоритма, созданного в рамках данной дипломной работы. %Расписываются действия производимые на каждом этапе.

В пятой главе производится описание созданного прототипа и архитектурные решения используемые при его создании.
% Описываются ключевые моменты реализации и построения интерфейсов.

В шестой главе производится оценка полученных результатов, приводятся графики и примеры.
%Приводится объяснение влияния каждого упомянутого фактора.

В седьмой главе производится оценка обеспечения безопасных условий труда при проведении исследования.

В восьмой главе производится технико-экономическое обоснование эффективности данной разработки.

В заключении подводятся итоги и делаются выводы по дипломной работе, а также описывается дальнейший план развития проекта.

\clearpage
