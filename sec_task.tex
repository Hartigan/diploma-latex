\chapter{Постановка задачи}
\label{sec:practice:planning}

Быстрая и лекго конструируемая обработка данных --- одна из задач, решаемых с использованием реактивного и функционального.

Исследования и разработки, проводимые в рамках диссертации, являются основой для анализа и проектирования новых способов обработки данных.
Данная работа предназначена использования в реальных проектах, в которых существует множество взаимодействий между модулями или пользователями.

\section{Постановка задачи}
\label{sub:practice:task_planning}

В рамках диссертации поставлена цель построить систему для обработки наборов данных.
Данные представляются в виде коллекций и списков.
Обработка должна осуществлятся с помощью операций схожих с операциями реляционной алгебры.
Должна быть построена абстрактная модель системы и реализована на одной из выбранных платформ.
Все реализованные операции следует протестировать на предмет ошибок.
Исследовать пути для оптимизации полученного решения.
Для решения поставленной задачи необходимо:
\begin{itemize}
  \item создать абстрактную модель исходных данных;
  \item создать протокол взаимодейтвия между частями системы;
  \item создать базовые модели для операций над данными;
  \item	определить список требуемых операций;
  \item	реализовать операции;
  \item	подготовить тесты.
\end{itemize}

\section{Определение границ исследования}
\label{sub:practice:task_milestone}

В рамках диссертации необходимо получить библиотеку для обработки наборов данных, основанную на событийной модели.
Для этого необходимо:
\begin{itemize}
  \item построить абстрактную модель системы;
  \item определить интерфейс библиотеки;
  \item протестировать по принципу черного ящика.
\end{itemize}

На следующем этапе необходимо разработать приложение с тестами.
Приложение будет работать с несколькими наборами исходных данных и реагировать на действия пользователя.
После разработки прототипа необходимо подготовить подробное описание его структуры и сделать выводы о его эффективности.
На основе проделанной работы сделать заключение о возможности использования в реальных приложениях.
