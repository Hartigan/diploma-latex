\section{Постановка задачи}
\label{sec:practice:planning}

Сжатие данных --- одна из задач, решаемых с использованием нейронных сетей.
Как и любое сжатие, решение данной задачи основано на устранении избыточности.

Способность нейронных сетей к выявлению взаимосвязей между различными параметрами дает возможность выразить данные большой размерности более компактно, если данные тесно взаимосвязаны друг с другом.

Обратный процесс – восстановление исходного набора данных из части информации --- называется ассоциативной памятью.
Ассоциативная память позволяет также восстанавливать исходный сигнал или образ из поврежденных входных данных.

Исследования и разработки, проводимые в рамках дипломной работы, являются основой для анализа и проектирования новых способов использования нейронных сетей в области сжатия данных.
Данная работа предназначена для оценки эффективности использования многослойных перцептронов для сжатия графической информации,
учитывая такие факторы как степень сжатия, качество полученного изображения на выходе,
а также вопрос производительности и нахождение оптимальных параметров нейронной сети.


\subsection{Постановка задачи}
\label{sub:practice:task_planning}

В рамках дипломной работы поставлена цель найти способ сжимать графические данные с использованием многослойных перцептронов.
Графическая информация представляется на компьютере как набор точек.
Каждая точка имеет такие характеристики как позиция относительно границ и цвет.
Процесс сжатия называется сжимающим кодированием. Сжатие основано на устранении избыточности, содержащейся в исходных данных.
Иными словами, для сжатия данных используются некоторые априорные сведения о том, какого рода данные сжимаются.
Не обладая такими сведениями об источнике, невозможно сделать никаких предположений о преобразовании, которое позволило бы уменьшить объём данных.
Модель избыточности может быть статической, неизменной для всего сжимаемого сообщения, либо строиться на этапе сжатия и восстановления.
Методы, позволяющие на основе входных данных изменять модель избыточности информации, называются адаптивными.
Неадаптивными являются обычно узкоспециализированные алгоритмы, применяемые для работы с данными, обладающими хорошо определёнными и неизменными характеристиками.
Нейронная сеть позволяет реализовать алгоритм сжатия графической информации методом сжатия с потерями.
Для решения поставленной задачи с помощью нейронных сетей необходимо:
\begin{itemize}
  \item собрать данные для обучения;
  \item подготовить и нормализовать данные;
  \item выбрать топологию сети;
  \item	эксперементально подобрать характеристики сети;
  \item	эксперементально подобрать параметры для обучения;
  \item	обучить сеть;
  \item	проверить адекватность обучения сети.
\end{itemize}

\subsection{Определение границ исследования}
\label{sub:practice:task_milestone}

В рамках исследования необходимо получить численные данные о результатах использования нейронных сетей для сжатия данных.
Для этого необходимо:
\begin{itemize}
  \item выбрать тестовые данные
  \item выбрать тип и топологию нейронной сети
  \item составить поэтапный алгоритм преобразования изображения
\end{itemize}

На следующем этапе необходимо разработать тестовое приложение.
Приложение будет работать по составленному алгоритму и собирать статистику.
Полученные данные можно представить в виде таблицы или графиков.

После разработки прототипа необходимо подготовить подробное описание его структуры и сделать выводы о его эффективности.
На основе проделанной работы сделать заключение о возможности использования нейронный сетей для сжатия графической информации.
