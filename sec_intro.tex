\sectioncentered*{Введение}
\addcontentsline{toc}{section}{Введение}
\label{sec:intro}

В последнее десятилетие наблюдается возросший интерес к нейронным сетям, которые успешно применяются в самых различных областях: медицина, физика, химия, системы безопасности.

Нейронные сети вошли в практику везде, где требуется решать задачи прогнозирования, классификации или управления.
Такие направления использования во многом обусловлены несколькими причинами: богатыми возможностями и простота в использовании.
Нейронные сети – исключительно мощный метод моделирования, позволяющий воспроизводить чрезвычайно сложные зависимости.
В частности, нейронные сети  являются нелинейными по своей природе.
На протяжении длительного времени линейное программирование было основным методом моделирования в большинстве областей, потому что для него хорошо разработаны процедуры оптимизации.
Следовательно, нейронные сети могут решать некоторые типы задач, с которыми возникают сложности при применении методов линейного программирования.

Нейронные сети учатся на примерах. Пользователь нейронной сети подбирает представительные данные, а затем запускает алгоритм обучения, который автоматически воспринимает структуру данных.
При этом от пользователя, требуется набор эвристических знания о том, как следует отбирать и подготавливать данные для обучения, подбирать архитектуру сети и правильно интерпретировать результаты.

Нейронные сети привлекательны с интуитивной точки зрения, ибо они основаны на примитивной биологической модели нервных систем. В будущем развитие таких нейробиологических моделей может привести к созданию действительно мыслящих компьютеров.

Целью дипломной работы является исследования возможностей использования нейронных сетей для сжатия графической информации, оценка качества полученных изображений, степени сжатия и скорости обработки данных.
