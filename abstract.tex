\sectioncentered*{Реферат}
\thispagestyle{empty}

\emph{Ключевые слова}: НЕЙРОННЫЕ СЕТИ; СЖАТИЕ С ПОТЕРЯМИ; ЧАСТОТА ДИСКРЕТИЗАЦИИ; КРИТЕРИИ СРАВНЕНИЯ ИЗОБРАЖЕНИЙ.

\vspace{4\parsep}

Дипломная работа выполнена на 6 листах формата А1 с пояснительной запиской на~\pageref*{LastPage} страницах, без приложений справочного или информационного характера.
Пояснительная записка включает \total{chapter}~глав, \totfig{}~рисунков, \tottab{}~таблиц, \toteq{}~формулы, \totref{}~литературный источник.

Целью дипломной работы является разработка алгоритма, пригодного для решения практических задач, возникающих в реальных проектах, связанных с обработкой и передачи изображений и другой мультимедийной информации.

Для достижения цели дипломной работы был разработан алгоритм, предназначенный для сжатия графической или любой другой мультимедийной информации.
Созданные с использованием данного алгоритма библиотеки могут быть использованы в реальных проектах, использующих методы сжатия с потерями для мультемидийной информации.
В работе приведена библиотека реализующая алгоритм на платформе \dotnet{}.

В разделе технико"=экономического обоснования был произведён расчёт затрат на разработку данного алгоритма, а также прибыли от разработки, получаемой разработчиком.
Проведённые расчёты показали экономическую целесообразность проведения исследования.

Пояснительная записка включает раздел по охране труда, в котором была произведена оценка безопасных условий труда на предприятии, где частично разрабатывалась данная дипломная работа.

\clearpage
