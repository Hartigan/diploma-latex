\section{Разработка алгоритма}
\label{sec:research:development_algotithm}

Идея, лежащая в основе всех алгоритмов сжатия с потерями, довольно проста:
\begin{itemize}
  \item на первом этапе удалить несущественную информацию;
  \item на втором этапе к оставшимся данным применить наиболее подходящий алгоритм сжатия без потерь.
\end{itemize}

Основные сложности заключаются в выделении этой несущественной информации.
Подходы здесь существенно различаются в зависимости от типа сжимаемых данных.
Для звука чаще всего удаляют частоты, которые человек просто не способен воспринять, уменьшают частоту дискретизации,
а также некоторые алгоритмы удаляют тихие звуки, следующие сразу за громкими, для видеоданных кодируют только движущиеся объекты,
а незначительные изменения на неподвижных объектах просто отбрасывают. Для одиночных изображений, как и для звука,
отсеиваются элементы которые человек не способен различить.
Методы выделения несущественной информации на изображениях будут подробно рассмотрены далее.

\subsection{Критерии качества сжатого изображения}
\label{sub:research:errors}

Прежде чем говорить об алгоритмах сжатиях с потерями, необходимо договориться о том,
что считать приемлемыми потерями. Понятно, что главным критерием остаётся визуальная оценка изображения,
но также изменения в сжатом изображении могут быть оценены количественно.
Самый простой способ оценки --- это вычисление непосредственной разности сжатого и исходного изображений.

\begin{equation}
  \label{eq:research:image_delta}
  error = \sum_{i=1}^{N}\sum_{j=1}^{M}\left | s_{i,j}-a_{i,j} \right |
\end{equation}
\begin{explanation}
где & $ s_{i,j} $ & значение пикселя в позиции i,j исходного изображения;\\
  & $ a_{i,j} $ & значение пикселя в позиции i,j сжатого изображения изображения;\\
  & $ N $ & высота изображения в пикселях;\\
  & $ M $ & ширина изображения в пикселях.
\end{explanation}

Очевидно, что, чем больше суммарная ошибка, тем сильнее искажения на сжатом изображении.
Тем не менее, эту величину крайне редко используют на практике, т.к. она никак не учитывает размеры изображения,
а значит не поможет оценить метод сжатия изображений разной величины.
Гораздо шире применяется оценка с использованием среднеквадратичного отклонения~\cite{quadratic_deviation}:

\begin{equation}
  \label{eq:research:image_standard_deviation}
  error = \sqrt{\frac{\sum_{i=1}^{N}\sum_{j=1}^{M}(s_{i,j}-a_{i,j})^{2}}{N*M}}
\end{equation}
\begin{explanation}
где & $ s_{i,j} $ & значение пикселя в позиции i,j исходного изображения;\\
  & $ a_{i,j} $ & значение пикселя в позиции i,j сжатого изображения изображения;\\
  & $ N $ & высота изображения в пикселях;\\
  & $ M $ & ширина изображения в пикселях.
\end{explanation}

Другой подход заключается в следующем: пиксели итогового изображения рассматриваются как сумма пикселей исходного изображения и шума.
Критерием качества при таком подходе называют величину отношения сигнал-шум (SNR), вычисляемую следующим образом:

\begin{equation}
  \label{eq:research:image_snr}
  SNR = \sqrt{\frac{\sum_{i=1}^{N}\sum_{j=1}^{M}a_{i,j}^{2}}{\sum_{i=1}^{N}\sum_{j=1}^{M}(s_{i,j}-a_{i,j})^{2}}}
\end{equation}
\begin{explanation}
где & $ s_{i,j} $ & значение пикселя в позиции i,j исходного изображения;\\
  & $ a_{i,j} $ & значение пикселя в позиции i,j сжатого изображения изображения;\\
  & $ N $ & высота изображения в пикселях;\\
  & $ M $ & ширина изображения в пикселях.
\end{explanation}

Две последние формулы будут использоваться для оценки качества полученного изображения.
